\begin{titlepage}
	\begin{center}
		\includegraphics[width=0.4\textwidth]{fileForWriting/BigCrest}\\ \vspace{15 mm}
		\textsc{\Large Year 2 Project}\\ \vspace{15 mm}
		\doublespace
		\HRule \\ \vspace{8 mm}
		{\huge \bfseries Wearable Sensor, or Digital Twin}       % <<<< Put your title here
		\\\vspace{4 mm}
		\HRule \\ \vspace{25 mm}

		Yixiao \textsc{Shen} (ID 201676945)      \\        % <<<< Your names
		Jiajun \textsc{Guo} (ID 201676348)      \\        % <<<< Your names
		Qi \textsc{Zhou} (ID 201677596)      \\        % <<<< Your names
		Charles \textsc{Canning} (ID 201578458)      \\        % <<<< Your names
		Jack \textsc{Brissenden} (ID 201587773)      \\        % <<<< Your names
		Group 2p23                                 \\        % <<<< Your group number

		\vspace{15mm}
		\emph{Supervised by } Dr Dave \textsc{McIntosh}     % <<<< Your supervisor
		\vfill             % Bottom of the page
		{\large \today}    % today's date
	\end{center}
\end{titlepage}

%%%%%%%%%%%%%%%%%%%%%%%%%%%%%%%%%%%%%%%%%%%%%%%%%%%%%%%%%%%%%%%%%%%%%%%%%%%%%%%%%%%%%%%%%%%%%%
%Abstract
%%%%%%%%%%%%%%%%%%%%%%%%%%%%%%%%%%%%%%%%%%%%%%%%%%%%%%%%%%%%%%%%%%

\begin{abstract}
\parindent
Previous research on avian flight dynamics has utilized optical motion capture systems for birds to collect their indoor wingbeat motion data for simulation and analysis. However, these systems are limited by site constraints. To overcome this limitation, recent studies have focused on installing single IMUs on birds to acquire basic wingbeat data without space limitations. Despite these efforts, there is still a significant research gap in developing a multi-IMU bio-logging device capable of comprehensively capturing the complete wingbeat motion of birds.
To address this deficiency, this project aims to design a simple wearable sensor system based on four IMUs for human walking pattern recognition. Additionally, a front-end displaying engine based on an external Java graphic library was also developed to create a digital representation of humans, which enables a motion-following functionality. Although the technology will be initially tested on humans, the technology could then be applicable to tracking the motion and behaviour of animals in order to learn from the elegant designs seen in the natural world.

\end{abstract}


%%%%%%%%%%%%%%%%%%%%%%%%%%%%%%%%%%%%%%%%%%%%%%%%%%%%%%%%%%%%%%%%%%%%%%%%%%%%%%%%%%%%%%%%%%%%%%
%Declaration
%%%%%%%%%%%%%%%%%%%%%%%%%%%%%%%%%%%%%%%%%%%%%%%%%%%%%%%%%%%%%%%%%%
\newpage
\rule{0mm}{30mm}
\begin{center}
	\textbf{Declaration}
\end{center}

\fbox{\parbox{0.92\textwidth}{I confirm that I have read and understood the University’s definitions of plagiarism and collusion from the Code of Practice on Assessment. I confirm that I have neither committed plagiarism in the completion of this work nor have I colluded with any other party in the preparation and production of this work. The work presented here is my own and in my own words except where I have clearly indicated and acknowledged that I have quoted or used figures from published or unpublished sources (including the web). I understand the consequences of engaging in plagiarism and collusion as described in the Code of Practice on Assessment (Appendix L).}}


%%%%%%%%%%%%%%%%%%%%%%%%%%%%%%%%%%%%%%%%%%%%%%%%%%%%%%%%%%%%%%%%%%%%%%%%%%%%%%%%%%%%%%%%%%%%%%
%Contents
%%%%%%%%%%%%%%%%%%%%%%%%%%%%%%%%%%%%%%%%%%%%%%%%%%%%%%%%%%%%%%%%%%
\newpage \tableofcontents
\newpage \listoffigures
\newpage \lstlistoflistings

\newpage \onehalfspace
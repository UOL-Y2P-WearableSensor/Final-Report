\chapter{Introduction}
Avian flight dynamics have been a topic of interest for many researchers over the years.
To study the indoor wingbeat action of birds, previous researches have utilized optical motion capture systems to collect data.
For example, in order to replicate the motions of avian flight in a physical simulation, a group of researchers from Korea employed a marker-based optical motion capture system and high-speed cameras to gather flight data from pigeons, as reported by Ju et al.
in 2011~\cite{ju_data-driven_2011}.
Using the observed data, they were able to develop more realistic wingbeat movements.

However, such optical bio-logging systems have practical limitations, such as site constraints and high costs.
Recent studies have attempted to address these challenges by attaching a single inertial measurement unit (IMU) to birds to collect basic wingbeat data.
For instance, in 2018, Taylor et al.
compared the flight behavior of pigeons flying alone to those flying in pairs, by attaching a global position system (GPS) tracker and a 200 Hz tri-axial accelerometer to each bird for data collection~\cite{taylor_birds_2019}.
The researchers found that paired individuals exhibited a higher frequency of wing flaps, suggesting a potential improvement in maneuverability and flight stability.

Nevertheless, the information obtained from a single IMU is limited and may not provide a comprehensive representation of a bird's complete wingbeat movement.
A fundamental research gap remains in the development of a multi-IMU bio-logging device capable of comprehensively capturing the entire wingbeat action of birds.

To fill such gap, this project aims to design a simple wearable sensor system based on four IMUs for human walking pattern recognition, capturing comprehensive motion data that can be used for analysis and simulation purposes.
Furthermore, a front-end displaying engine will also be created to generate a digital twin of humans in real time.
This will not only enable motion tracking capabilities, but also allow for easy monitoring through a web browser with rapid deployment.

While the initial testing of the system will be conducted on human subjects, the potential applicability in tracking the movement and behavior of birds is promising.
The data collected by this device would provide a valuable opportunity to gain insights from the intricate and sophisticated designs found in nature.
For instance, one potential application is to study the flight patterns of birds and how they utilize their wings to achieve optimal efficiency, especially when faced with varying wind directions.

In conclusion, this project aims to develop a cost-effective and easy-to-use multi-IMU bio-logging device that can comprehensively capture the complete motion of objects.
The proposed device has the potential to contribute not only to the research on avian flight dynamics but also to the broader field of zoology, which may serve as a useful tool for researchers in various fields.


\section{Objectives}
The primary goal of this research project is to create a digital twin of a human using wearable sensors.
To achieve such target, the project has been divided into the following three sub-objectives:
\begin{enumerate}\label{obj}
    \item \label{itm:obj-data-collection}Data collection:
    Design a wearable device capable of collecting lower body motion data by attaching four IMUs, two on each femur and tibia.
    To enable unrestricted movement without the need for cables, the device will feature wireless communication capabilities.
    In addition, the device will integrate an independent power source to ensure uninterrupted data collection during testing.

    \item \label{itm:obj-data-integration}Data integration:
    A simple C\texttt{++} socket server will be developed to serve as the central hub for collecting and processing the motion data collected by the IMUs.
    This server will also be responsible for transmitting files to the front-end.

    \item \label{itm:obj-model-creation} Model creation:
    To enable the core functionality of motion-following, a simple front-end displaying engine will be developed to create a model of the human body.
    This model should be input with the motion data collected by the IMUs to replicate corresponding movements.
\end{enumerate}


